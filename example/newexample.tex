\documentclass[
        handout,
        %draft,
        ]{beamer}

% math
\usepackage{amssymb,latexsym,amssymb,amsmath,amsbsy,amsopn,amstext,upgreek}

% pictures, colors and multicolumn
\usepackage{color,multicol}
\usepackage{graphicx,wrapfig,fancybox,watermark,graphics}
\usepackage{picins}
\usepackage{pgf}
\usepackage{media9}

% pdf options
\usepackage{hyperref}
\usepackage{pdfpages}

% algorithms
\usepackage{listings,bera}
\definecolor{keywords}{RGB}{255,0,90}
\definecolor{comments}{RGB}{60,179,113}
\lstset{language=C,
keywordstyle=\color{keywords},
commentstyle=\color{comments}\emph}
\hypersetup{
    pdfpagemode=FullScreen, % show in full screen?
}
\usepackage{algorithm}
\usepackage{algorithmic}
\renewcommand{\algorithmicrequire}{\textbf{Input:}}
\renewcommand{\algorithmicensure}{\textbf{Output:}}

% reference entry
\usepackage{bibentry, natbib}
% reference style
\bibliographystyle{IEEEtran} 
%reference lib
\nobibliography{refs}

% include pdf page
\newcommand{\inpdfu}[2]{\begin{figure}\centering\includegraphics[trim=1.4in 6.5in 1.4in 1.4in, clip, height=0.7\textheight, page={#2}]{im/lecture_#1}\end{figure}}
\newcommand{\inpdfl}[2]{\begin{figure}\centering\includegraphics[trim=1.4in 1.4in 1.4in 6.5in, clip, height=0.7\textheight, page={#2}]{im/lecture_#1}\end{figure}}
\newcommand{\inpdfc}[2]{\begin{figure}\centering\includegraphics[trim=0.15in 0.2in 0.15in 0.2in, clip, height=0.75\textheight, page={#2}]{im/lecture_#1}\end{figure}}

% include png images
\newcommand{\inpng}[1]{\begin{figure}\centering\includegraphics[height=0.55\textheight]{im/#1}\end{figure}}

\usepackage[
	%compress,
	%minimal,
	%nonav,
	%red,
	%gold,
    blue,
	numbers,
	%nologo,
	polyu,
    comp,
    %forty,
    %seventyfive,
	]{beamerthemeHongKong}

% fonts
\usepackage{xeCJK}
\usepackage{xltxtra}
%       beamer
% \usefonttheme{default} % sans serif
\usefonttheme{professionalfonts}
\usefonttheme{serif}
% \usefonttheme{structurebold}
% \usefonttheme{structureitalicserif}
% \usefonttheme{strucutresmallcapsserif}
%       English
\setmainfont{Times New Roman}
\setsansfont{Arial}
\setmonofont{Inconsolata}
%       Chinese
\setCJKmainfont[BoldFont={Hiragino Sans GB W6},ItalicFont={方正启体简体}]{方正新书宋简体}
\setCJKsansfont{Hiragino Sans GB W3}
\setCJKmonofont{文泉驿等宽正黑}

%%%%%%%%%%%%%%%%%%%%%%%%%% Title Page %%%%%%%%%%%%%%%%%%%%%%%%%%%%%%%%%%%%%%%%%%%%%%%%%%%%%%%%%

\title[Title 报告标题]{Report Template 报告模板}
\author[Author 作者]{Xiaofeng QU\texorpdfstring{曲晓峰, Research Student\\\tiny{csxfqu@comp.polyu.edu.hk}}{}}
\institute[Institute 机构]{Department of \textit{Computing} \textit{电子计算}学系\\\textit{The Hong Kong Polytechnic} University \textit{香港理工}大学}
\date{\today}

%%%%%%%%%%%%%%%%%%%%%%%%%% Document %%%%%%%%%%%%%%%%%%%%%%%%%%%%%%%%%%%%%%%%%%%%%%%%%%%%%%%%%%%

\begin{document}

\frame{\titlepage}

\section*{Table of Contents}
\begin{frame}{\secname}
    \tableofcontents
\end{frame}

\AtBeginSubsection[] {
    \begin{frame}<handout:0>{Outline}
        \tableofcontents[current,currentsubsection]
    \end{frame}
}


\section{Tips of Beamer}

\subsection{\secname}
\begin{frame}[t]{\subsecname}{subtitle} % t-top, c-center, s-shrank
    \begin{enumerate}[<+-|alert@+>]
    \item Use \texttt{[<+-|alert@+>]} in the \texttt{enumerate} or \texttt{itemize} environments to generate multiple pages with the items showing up and alerted one by one.
    \item Don't use \texttt{\textbackslash overprint}. It messes up the the PDF page number, \texttt{draft} and \texttt{handout} options.
    \item Use \text{\textbackslash note} to add notes to the handout.
    \end{enumerate}
\end{frame}

\subsection{Block Environments}
\begin{frame}[t]{\subsecname} % t-top, c-center, s-shrank
    \begin{columns}
    \column{0.5\textwidth}
        \begin{block}{This is a Block}
            This is important information.
        \end{block}
        \begin{alertblock}{This is an Alert block}<2->
            This is important alert.
        \end{alertblock}
        \begin{exampleblock}{This is an Example block}<3->
            This is an example.
        \end{exampleblock}
        \begin{definition}<4->
            A prime number is a number that \ldots
        \end{definition}
    \column{0.5\textwidth}
        \begin{example}<5->
            This is an example of \ldots
        \end{example}
        \begin{theorem}<6->
            $ a^2 + b^2 = c^2 $
        \end{theorem}
        \begin{corollary}<7->
            $ x + y = y + x $
        \end{corollary}
        \begin{proof}<8->
            $\omega +\phi = \epsilon $
        \end{proof}
    \end{columns}
\end{frame}

\subsection{Including Code}
\begin{frame}[fragile]{\subsecname with a \texttt{fragile} option} % t-top, c-center, s-shrank
    \begin{semiverbatim}
\\begin\{frame\}
\\frametitle\{Outline\}
\\tableofcontents
\\end\{frame\}
    \end{semiverbatim}
\end{frame}

\subsection{测试中文字体}
\begin{frame}[t]{\subsecname} % t-top, c-center, s-shrank
    \begin{itemize}
    \item \textrm{正文字体,建议用宋体}
    \item \textit{斜体,建议用楷体}
    \item \textbf{粗体,建议用黑体}
    \item \textsf{对应西文无衬线字体,建议用黑体或者幼圆体}
    \item \texttt{对应西文等宽字体,建议用中文等宽字体}
    \end{itemize}
\end{frame}


\section{Layout}

\subsection{Two Images}
\begin{frame}[t]{\subsecname} % t-top, c-center, s-shrank
    \begin{columns}
    \column{0.5\textwidth}
        \begin{figure}
        \centering
        \includegraphics[width=\textwidth]{im/testim}
        \caption{Test Figure}
        \end{figure}
    \column{0.5\textwidth}
        \begin{figure}
        \centering
        \includegraphics[width=\textwidth]{im/testim}
        \caption{Test Figure}
        \end{figure}
    \end{columns}
\end{frame}

\subsection{Image and Text}
\begin{frame}[t]{\subsecname} % t-top, c-center, s-shrank
    \begin{columns}
    \column{0.5\textwidth}
        \begin{figure}
        \centering
        \includegraphics[width=\textwidth]{im/testim}
        \caption{Test Figure}
        \end{figure}
    \column{0.5\textwidth}
        Use the \texttt{columns} environment to display the image and text in the same frame side by side vertically.
    \end{columns}
\end{frame}

\begin{frame}[t]{\subsecname} % t-top, c-center, s-shrank
    \begin{columns}
    \column{0.5\textwidth}
        Use the \texttt{columns} environment to display the image and text in the same frame side by side vertically.
    \column{0.5\textwidth}
        \begin{figure}
        \centering
        \includegraphics[width=\textwidth]{im/testim}
        \caption{Test Figure}
        \end{figure}
    \end{columns}
\end{frame}

\subsection{Don't be overcrowded}
\begin{frame}[t]{\subsecname} % t-top, c-center, s-shrank
    \begin{enumerate}
    \item Don't be overcrowded.
    \item Don't be overcrowded.
    \item Don't be overcrowded.
    \item Don't be overcrowded.
    \item Don't be overcrowded.

    \item Don't be overcrowded.
    \item Don't be overcrowded.
    \item Don't be overcrowded.
    \item Don't be overcrowded.
    \item Don't be overcrowded.
    
    \item Don't be overcrowded.
    \item Don't be overcrowded.
    \item Don't be overcrowded.
    \item Don't be overcrowded.
    \item Don't be overcrowded.
    
    \item Don't be overcrowded.
    \end{enumerate}
\end{frame}

\begin{frame}[shrink]{The \texttt{shrink}ed Frame} % t-top, c-center, shrink
    \begin{enumerate}
    \item \texttt{shrink} buys you a little space, but is not recommended.
    \item Don't be overcrowded.
    \item Don't be overcrowded.
    \item Don't be overcrowded.
    \item Don't be overcrowded.

    \item Don't be overcrowded.
    \item Don't be overcrowded.
    \item Don't be overcrowded.
    \item Don't be overcrowded.
    \item Don't be overcrowded.
    
    \item Don't be overcrowded.
    \item Don't be overcrowded.
    \item Don't be overcrowded.
    \item Don't be overcrowded.
    \item Don't be overcrowded.
    
    \item Don't be overcrowded.
    \item Don't be overcrowded.
    \item Don't be overcrowded.
    \item Don't be overcrowded.
    \item Don't be overcrowded.
    \end{enumerate}
\end{frame}

\begin{frame}[allowframebreaks]{The \texttt{break}ed Frame} % t-top, c-center, shrink
    \begin{enumerate}
    \item \texttt{allowframebreaks} breaks a frame into several frames.
    \item Don't be overcrowded.
    \item Don't be overcrowded.
    \item Don't be overcrowded.
    \item Don't be overcrowded.

    \item Don't be overcrowded.
    \item Don't be overcrowded.
    \item Don't be overcrowded.
    \item Don't be overcrowded.
    \item Don't be overcrowded.
    
    \item Don't be overcrowded.
    \item Don't be overcrowded.
    \item Don't be overcrowded.
    \item Don't be overcrowded.
    \item Don't be overcrowded.
    
    \item Don't be overcrowded.
    \item Don't be overcrowded.
    \item Don't be overcrowded.
    \item Don't be overcrowded.
    \item Don't be overcrowded.
    \end{enumerate}
\end{frame}

%%%%%%%%%%%%%%%%%%%%%%%%%% Ending %%%%%%%%%%%%%%%%%%%%%%%%%%%%%%%%%%%%%%%%%%%%%%%%%%%%%%%%%%%%%

\begin{frame}<handout:0>[c]{Q \& A}
    \centerline{\LARGE{Any questions?\footnote{\texttt{<handout:0>} option removes this frame from \texttt{handout}.}}}    
\end{frame}    
    
\begin{frame}<handout:0>[c]{Acknowledgments}
    \centerline{\LARGE{Thank You For Your Attention!\footnote{Don't forget to Acknowledge.}}} 
\end{frame}
    
    
\end{document}



